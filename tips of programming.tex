\documentclass[onecolumn]{article}
\usepackage[UTF8]{ctex}
\usepackage{indentfirst}
\usepackage{amsmath}
\usepackage{amsfonts}
\usepackage{graphicx}
\usepackage{url}
\usepackage{color,xcolor}
\usepackage{algorithmicx}
\usepackage{listings}
\usepackage[a4paper,left=10mm,right=10mm,top=15mm,bottom=15mm]{geometry}
\setlength{\parindent}{2em}
\title{编程技巧}
\author{陈元昊}
\date{\today}
\begin{document}
\maketitle
\renewcommand{\contentsname}{目录}
\tableofcontents
\section{Linux}
    \subsection{指令技巧}
        \subsubsection{grep}
            \noindent
            \begin{enumerate}
                \item 加-E选项后,可以在使用正则匹配时不用给括号转义
            \end{enumerate}
        \subsubsection{gcc/g++}
            \noindent
            \begin{enumerate}
                \item 加-E选项后仅执行到预处理,文件后缀.i
                \item 加-S选项后仅执行到编译,文件后缀.s
                \item 加-C选项后仅执行到汇编,文件后缀.o
            \end{enumerate}
        \subsubsection{git}
            \noindent
            \begin{enumerate}
                \item 当因为token的原因(一般存在于报错)无法clone时,可尝试设置一个在网站上设置一个具有权限的token,复制之,然后在Windows凭据管理器上新建/修改一个普通凭据,注意密码应为token
            \end{enumerate}    
    \subsection{基本概念}
        \subsubsection{挂起与后台运行}
            \noindent
            \begin{enumerate}
                \item 挂起,一般通过按ctrl+z实现,效果为暂停执行(前台或后台程序均可以),但可用fg或bg恢复执行
                \item 后台运行,一般通过在命令行末尾加“\&”符号实现,也可以通过挂起+后台恢复间接实现,效果为以不占用终端的方式运行
            \end{enumerate}
    \subsection{WSL2}
        \subsubsection{网络}
            \noindent
            \begin{enumerate}
                \item 宿主机可以用127.0.0.1访问WSL2,反之则不行
                \item 当代理软件(Clash)位于Windows上时,Windows配置代理仅需要set http(s)\_proxy="127.0.0.1:7890",而WSL2在使用export http(s)\_proxy="宿主机IP:7890"之前,要先用cat /etc/resolv.conf | grep nameserver | awk '{ print \$2 }'获取宿主机IP(此外,用hostname -I | awk '{print \$1}'获取WSL2自身IP)
            \end{enumerate}
\newpage
\section{C/C++}
    \subsection{算法技巧}
        \subsubsection{预处理}
            \noindent
            \begin{enumerate}
                \item 二分前先使数组有序
                \item 注意隐藏边界(长度为0,1等)
                \item 先排序再计算往往可以简化计算过程
                \item 有可能样例输入有序,测试点输入无序
                \item 注意图的输入中的重边和自环,以及有向输入转化为无向图
            \end{enumerate}
        \subsubsection{算法执行}
            \noindent
            \begin{enumerate}
                \item 二分区间的开闭由具体问题决定(一般一边开一边闭)
                \item 递归算法需要数组记录答案时可以不用“触底”时全部修改,然后利用一个全局的bool变量连续退出,而是可以回溯时逐步修改,从而减小代码复杂度
                \item 注意浮点数计算的上下浮动
                \item 尝试将$n!$(排列)转化为$2^n$(组合)
            \end{enumerate}
        \subsubsection{算法评估}
            \noindent
            \begin{enumerate}
                \item 计算递归算法复杂度可先计算递归实例的数量
            \end{enumerate}
    \subsection{语言特性}
        \subsubsection{基本语法}
            \noindent
            \begin{enumerate}
                \item 注意循环嵌套中,循环变量i、j、k等不要重复使用
                \item 循环体中的变量地址不变
                \item 使用getchar前注意去除cin等留下的回车等干扰字符
                \item 注意数组下标越界有可能完全无异常(越在其他变量内部)
                \item switch分支结构注意用break
                \item 函数调用计数可利用函数体中的局部静态变量
                \item 可以使用位域直接操作内存中的位
            \end{enumerate}
        \subsubsection{面向对象}
            \noindent
            \begin{enumerate}
                \item 对象内部局部变量需要初始化
                \item 注意写public(默认为private)
                \item 友元函数函数不是成员函数,不能加作用域符号
                \item 引用本质只是别名,其创建时不会产生任何构造过程
                \item 注意避免自身赋值
                \item 当一个内部类或内部对象需要访问外部对象时,尽量通过外部对象成员变量的指针来访问,否则有可能出现构造顺序或访问权限的问题
                \item 尽量不要创建野指针,如果不可避免要创建野指针,一定要初始化为nullptr
                \item 移动构造、赋值前注意删除当前指针的内容,避免当前指针赋新值后内存泄漏
                \item delete前对象最好指针最好不是nullptr,delete后对象指针最好置为nullptr
                \item 在返回值和参数均可被析构时,先析构返回值,再析构参数(符合栈的顺序)
                \item 虚函数/常量函数不能为静态函数,因为其调用/参数中需要/含有this指针
                \item 模板函数将成员函数作为形参时,成员函数应设为静态函数,非静态成员函数因为有this指针形参,参数数量不一致,可能导致错误(sort)
                \item std::move()对常引用无效
                \item 派生类新定义的非虚函数和新定义的变量会在函数形参为值/引用/指针(所有情况)时被切片
                \item 重写函数调用时,与所有当前形式类中的函数同名且参数不同的函数会被隐藏,然后按虚函数机制调用
                \item 在派生类没有直接写出新函数的情况下,派生类不会自动生成新的虚函数继承版本,而是在虚函数表中沿用旧版本(注意与重写隐藏的关系)
                \item 基类指针指向派生类对象时,调用被基类声明、派生类继承的虚函数不需要dynamic\_cast,调用派生类声明的函数需要dynamic\_cast
                \item 模板的声明与实现需要在同一文件中(模板实例化在编译期确定)
            \end{enumerate}
    \subsection{Qt}
        \subsubsection{绘图}
            \noindent
            \begin{enumerate}
                \item Qt画圆的坐标原点为外界矩形的左上角点
            \end{enumerate}
        \subsubsection{原理}
            \noindent
            \begin{enumerate}
                \item Qt画圆的坐标原点为外界矩形的左上角点
            \end{enumerate}        
\newpage
\section{Python}
    \subsection{语言规范}
        \subsubsection{程序结构}
            \noindent
            \begin{enumerate}
                \item 引用原生库和手写库
                \item 定义全局变量
                \item 定义修饰器
                \item 定义类(包括函数对象)
                \item 定义函数(包括argparse)
                \item 定义主函数
            \end{enumerate}
    \subsection{语言特性}
        \subsubsection{运行特性}
            \noindent
            \begin{enumerate}
                \item 在使用import时正确的路径是针对main.py而言的,而不是针对当前文件而言的
                \item global关键字的使用是为了在局部作用域中引用并修改全局变量
                \item 闭包函数若要修改上级作用域中的变量,需要用nonlocal关键字
                \item Python中只有模块(module),类(class)以及函数(def、lambda)才会引入新的作用域,其它的代码块(如if、try、for等)不会引入新的作用域,因此在代码块外部可以直接引用代码块内声明的变量
                \item Python的变量是动态声明的,未考虑到这点可能出现bug,例如在if语句中声明了变量,若该if语句条件为假,则不仅其内部语句不执行,其内部变量也不会被定义。因此,很多时候有必要在if语句之前声明变量
            \end{enumerate}
    \subsection{具体应用}
        \subsubsection{正则表达式}
            \noindent
            \begin{enumerate}
                \item 在?、+、*以及\{n,m\}后加?表示进行懒惰匹配(与默认的贪婪匹配相反)
                \item \textbackslash b、\textdollar以及\^{}匹配的是单词边界,而非字符(匹配的是“一条线”)
            \end{enumerate}
\newpage
\section{Javascript}
    \subsection{语言特性}
        \subsubsection{函数}
            \noindent
            \begin{enumerate}
                \item 闭包中的作用于整个函数的变量为引用,而在某个循环内部的变量为拷贝
            \end{enumerate}
        \subsubsection{对象与原型}
            \noindent
            \begin{enumerate}
                \item this永远指向最近的调用者
            \end{enumerate}        
\end{document}